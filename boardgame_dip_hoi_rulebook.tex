\documentclass{jsarticle}
\usepackage[utf8]{inputenc}
\usepackage[dvipdfmx]{graphicx}
\usepackage{amssymb, amsmath,latexsym,mathtools}
\usepackage{listings,jvlisting} 
\usepackage[margin=12truemm]{geometry}
\renewcommand{\baselinestretch}{1.5}
\title{diplomacyとhoi4と主計将校のパクリゲーのルール}
\author{classicarbon}
\begin{document}
\date{}
\maketitle
\section{草案}
\subsection{欲しい要素}
\noindent
街(半年で建設、5個消費)\\
物資(一つの街で3個生産、軍隊の生産に2個消費、軍隊の維持に1個消費)\\
補給所(半年で建設、3個消費)\\
補給(補給所がある地点は軍隊+2、その周辺は+1。補給不可能な場合その区域の軍隊は全て行動不能となる。)\\
再配置(軍隊めっちゃ高速移動、1個消費で補給所を利用した経路が通っている場所に移動可能)\\
包囲(周りを別の国の軍隊に囲まれること。都市の場合生産が無くなる/補給所の場合補給能力が無くなる)\\
空軍(どうしようねこれ。制空攻撃/近接支援/戦略爆撃/空挺輸送。後ろ三つは制空状態でないと使用不可能)\\
資源輸送(空軍と海軍が出来たら熱いねとか思っている。それしか思っていない)\\
軍隊の解体(包囲されていない場合は物資1が返ってくる。)\\
戦闘ルール(敵軍の1.5倍以上の軍隊で攻めた場合移動先の地域を確保できる/保守の場合1.5倍の敵までは耐えられる。また、2倍より大きい数の敵から攻撃を受けた場合、その超過分だけ軍隊を解体する必要がある。)\\

\subsection{陸軍の可能な命令}
陸軍は陸上でのみ行動可能
\begin{itemize}
  \item 保守
  \item 移動
  \item 輸送
  \item 再配置
  \item 支援
\end{itemize}

\subsubsection{保守}
現在地の保全を行う。行動命令が無かった場合自動で保守が選択される。\\
保守している地域で戦闘が起こった場合、そのプレイヤーは「その地域にいる自国軍の数」と「その地域に攻撃してきた他国軍の数」の少ない分だけ物資を支払う必要がある。

\subsubsection{移動}
軍隊の移動を行う。移動先の地域に他国軍がいる場合や、他国軍が移動してきた場合は戦闘が発生する。そのとき、プレイヤーは「その地域に移動した自国軍」の数だけ物資を支払う必要がある。

\subsubsection{輸送}
海軍や空軍によって輸送されることができる。移動範囲が広大であること以外は移動と同じである。

\subsubsection{再配置}
軍隊の大規模な移動を行う。補給所による補給が通っている区域内であれば、任意の場所に移動することができる。このとき、プレイヤーは経路を指定する必要がある。もしもその経路上で攻撃を受けた場合、その軍隊は手前で停止する。戦闘は起こらない。また、プレイヤーは物資1を支払う必要がある。

\subsubsection{支援}
隣接している地域の支援を行う。保守命令や移動命令が行われている区域において、支援を行うことができる。このとき、隣接している必要があるのは移動先や保守中の地域のみである。
\end{document}