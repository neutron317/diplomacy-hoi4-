\documentclass{jsarticle}
\usepackage[utf8]{inputenc}
\usepackage[dvipdfmx]{graphicx}
\usepackage{amssymb, amsmath,latexsym,mathtools}
\usepackage{listings,jvlisting} 
\usepackage[margin=12truemm]{geometry}
\renewcommand{\baselinestretch}{1.0}
\title{diplomacyとhoi4と主計将校のパクリゲーのルール}
\author{classicarbon}
\begin{document}
\date{}
\maketitle
\section{草案}
外交を含まない国家戦略も楽しめるゲームにしたいね
\subsection{欲しい要素}
\noindent
街(半年で建設、5個消費)\\
物資(一つの街で2個生産、軍隊の生産に2個消費、軍隊の維持に1個消費)\\
補給所(半年で建設、2個消費)\\
補給(補給所がある地点は軍隊+2、その周辺は+1。補給不可能な場合その区域の軍隊は全て行動不能となる。海軍も同様だが当然海上に補給所は建てられない)\\
再配置(軍隊めっちゃ高速移動、1個消費で補給所を利用した経路が通っている場所に移動可能)\\
包囲(周りを別の国の軍隊に囲まれること。都市の場合生産が無くなる/補給所の場合補給能力が無くなる)\\
空軍(どうしようねこれ。制空攻撃/近接支援/戦略爆撃/空挺輸送。後ろ三つは制空状態でないと使用不可能。要素が多すぎるから選択ルール扱い?)\\
資源輸送(空軍と海軍が出来たら熱いねとか思っている。それしか思っていない)\\
軍隊の解体(包囲されていない場合は物資1が返ってくる。)\\
戦闘ルール(敵軍の1.5倍以上の軍隊で攻めた場合移動先の地域を確保できる/保守の場合1.5倍の敵までは耐えられる。また、2倍より大きい数の敵から攻撃を受けた場合、その超過分だけ軍隊を解体する必要がある。)\\
国家命令(国家の方針として行う命令。都市の命令と言い換えても良いかも?)\\

\subsection{陸軍の可能な命令}
陸軍は陸上でのみ行動可能
\begin{itemize}
  \item 保守
  \item 移動
  \item 被輸送
  \item 再配置
  \item 支援
\end{itemize}

\subsubsection{保守}
現在地の保全を行う。行動命令が無かった場合自動で保守が選択される。\\
保守している地域で戦闘が起こった場合、そのプレイヤーは「その地域にいる自国軍の数」と「その地域に攻撃してきた他国軍の数」の少ない分だけ物資を支払う必要がある。

\subsubsection{移動}
軍隊の移動を行う。移動先の地域に他国の陸海軍がいる場合や、他国の陸海軍が同じ地域に移動してきた場合は戦闘が発生する。そのとき、プレイヤーは「その地域に移動した自国軍」の数だけ物資を支払う必要がある。

\subsubsection{被輸送}
海軍や空軍によって輸送されることができる。移動範囲が広大であること以外は移動と同じである。

\subsubsection{再配置}
軍隊の大規模な移動を行う。そのとき、プレイヤーは物資1を支払う必要がある。補給所による補給が通っている区域内であれば、任意の場所に移動することができる。このとき、プレイヤーは経路を指定する必要がある。もしもその経路上で攻撃を受けた場合、その軍隊は手前で停止する。戦闘は起こらない。

\subsubsection{支援}
隣接している地域を目標としている陸軍や海軍の支援を行う。保守命令や移動命令が行われている区域において、支援を行うことができる。このとき、隣接している必要があるのは移動先や保守中の地域のみである。\\
支援中の陸軍が存在する地域で戦闘が起こった場合、支援命令は取り消され保守命令へと変更される。

\subsection{海軍の可能な命令}
海軍は海岸及び海上でのみ行動可能
\begin{itemize}
  \item 保守
  \item 移動
  \item 輸送
  \item 支援
\end{itemize}
保守、移動、支援に関しては陸軍と同様である。

\subsubsection{保守}
陸軍と同様の命令。\\
現在地の保全を行う。行動命令が無かった場合自動で保守が選択される。\\
保守している地域で戦闘が起こった場合、そのプレイヤーは「その地域にいる自国軍の数」と「その地域に攻撃してきた他国軍の数」の少ない分だけ物資を支払う必要がある。

\subsubsection{移動}
陸軍と同様の命令。\\
軍隊の移動を行う。移動先の地域に他国の陸海軍がいる場合や、他国の陸海軍が同じ地域に移動してきた場合は戦闘が発生する。そのとき、プレイヤーは「その地域に移動した自国軍」の数だけ物資を支払う必要がある。

\subsubsection{輸送}
同じ地域に存在している陸軍や、同じ地域に存在している使われていない物資を輸送することができる。\\
プレイヤーは、輸送したい対象とその輸送経路を命令書に記す。この命令は、輸送に参加する海軍で同じ内容でなければならない。また、輸送は海軍同士が隣接している地域を経路に取ることができる。ただし、輸送中1区画のみ移動することができる。その際、移動する海軍はその旨を記載しなければならない。他の海軍がそれを命令書に記載する必要はない。また、輸送先は最終地点にいる海軍と同じ地域(当然陸上に限る)か、その隣接地域のどちらかを選択することができる。ただし、隣接地域を目的地とした場合、最後の海軍は移動することができない。\\
最初の海軍と同じ地域に存在している陸軍ではなく、その隣接地域に存在している陸軍を対象に選ぶこともできる。ただし、その場合は最初の海軍は移動することができない。また、同一経路を取るならば輸送できる量に限りはない。陸軍と物資を同時に輸送することもできる。\\
輸送中の海軍が戦闘状態に入った場合、輸送は取り消され保守命令へと変更される。ただし、最後の海軍が輸送先にて戦闘に入った場合、もしくは輸送されている陸軍が目的地で戦闘に入った場合はその地域に限り通常の戦闘として扱われ、輸送命令は解除されない。その戦闘の結果陸軍や海軍が撤退することになった場合、それぞれの軍隊は元いた地域に戻る。\\

\subsubsection{支援}
陸軍と同様の命令。\\
隣接している地域を目標としている陸軍や海軍の支援を行う。保守命令や移動命令が行われている区域において、支援を行うことができる。このとき、隣接している必要があるのは移動先や保守中の地域のみである。\\
支援中の海軍が存在する地域で戦闘が起こった場合、支援命令は取り消され保守命令へと変更される。

\subsection{空軍の可能な命令}
空軍は自国が占拠する補給所及びその周辺1区域、または海軍が存在している区域及びその周辺1区域で行動可能。また、一つの補給所につき存在できる空軍は4個まで。一つの海軍につき存在できる空軍は2個まで。区域が重複している場合、都合の良いように解釈できる。\\
空軍は、全ての命令において存在可能区域を2区域まで移動できる。\\
\begin{itemize}
  \item 制空
  \item 近接支援
  \item 輸送
  \item 爆撃
  \item 再配置
\end{itemize}
各命令は、移動の後に実行される。\\
2区域の移動の途中の区域で他国の空軍と存在区域が同一のものとなる場合、命令はその区域における制空に変更される。\\

\subsubsection{制空}
現在地の制空権の奪取を行う。命令が記載されていなかった場合、制空命令を受けた扱いとなる。\\
基本的には陸海軍の戦闘と同様だが、空軍の戦闘は制空を命令された空軍同士のみで行われる。この戦闘が終わった後他国の空軍がその区域に存在していなければ、その区域の制空権を奪取した扱いになる。\\

\subsubsection{近接支援}
陸海軍の支援が行える。ただし、支援が可能な区域は空軍が存在している区域のみである。\\
その区域の制空権を奪取できていない場合、支援することはできない。空軍の戦闘にも参加しない。\\

\subsubsection{輸送}
同じ地域に存在している陸軍や、同じ地域に存在している使われていない物資を輸送することができる。\\
プレイヤーは、輸送したい対象とその輸送経路を命令書に記す。この輸送は海軍のそれとは違い、一つの空軍しか参加できず、輸送できるのも一つの陸軍のみである。輸送箇所には他国の軍隊が存在している場所を選ぶこともできる。その軍隊が保守命令を選択しており、更にその戦闘で敵軍を撤退させることができなかった場合、輸送された陸軍は解体される。他国の軍隊が輸送先の区域に移動してきた場合、輸送された陸軍は撤退可能であれば撤退できる。\\
その区域の制空権を奪取できていない場合、輸送することはできない。空軍の戦闘にも参加しない。輸送が正常に実行されなかった場合、陸軍は元の箇所へと戻る。\\

\subsubsection{爆撃}
都市や補給所に対し爆撃が可能。一度爆撃を受けた都市や補給所は故障状態となる。\\
その区域の制空権を奪取できていない場合、爆撃することはできない。空軍の戦闘にも参加しない。爆撃が実行された場合、物資1を支払う必要がある。\\

\subsubsection{再配置}
軍隊の大規模な移動を行う。そのとき、プレイヤーは物資1を支払う必要がある。存在可能区域であれば、任意の場所に移動することができる。このとき、プレイヤーは経路を指定する必要がある。もしもその経路上で攻撃を受けた場合、その軍隊は手前で停止する。戦闘は起こらない。

\subsection{国家命令}
国家が軍隊の配置に関わらず実行可能な命令。基本的にすべての命令において物資を消費する。\\
\begin{itemize}
  \item 編成
  \item 解散
  \item 建設
  \item 解体
  \item 遷都
  \item 輸出
\end{itemize}
各命令は、軍隊の行動が終わった後に実行される。\\

\subsubsection{編成}
新たな軍隊を編成する。このとき、プレイヤーは自国の占領状態に無いいずれかの主要都市と、編成したい軍隊の種類を選択する。海岸でなければ海軍を編成することはできず、更に補給の要件を満たしていなければ編成することはできない。また、この都市が戦闘状態に入った場合や占領された場合は命令は取り消される。\\
命令が実行された場合物資2を消費する。この命令は1ターンで占領状態に無い主要都市の個数分行える。

\subsubsection{解体}
軍隊を解体する。ユニットを盤面から除去し、そのユニットが包囲状態に無い場合物資1を得る。軍隊維持のための物資消費に、この軍隊は含まれない。\\
この命令は1ターンで何回でも行える。

\subsubsection{建設}
あらたな都市や補給所を建設する。このとき、プレイヤーはターンの開始時に自国の軍隊によって占領されていた地域と、建設したいものを指定する。この時、指定した地域がターン終了時も自国によって占領されている必要は無いが、その地域が他国に占領された場合、もしくは戦闘状態に入った場合は命令は取り消される。また、既に何かしらの建物がある地域に更に建設することはできない。\\
命令が実行された場合建設するものが都市であれば物資5を、補給所であれば物資2を消費する。\\
また、故障状態にある建物を対象に選ぶこともできる。このとき、通常の建設と同じようにターン開始時に修理したい建物がある地域を自国の軍隊が占領している必要がある。\\
命令が実行された場合修理する者が都市であれば物資3を、補給所であれば物資1を消費する。\\
この命令は1ターンで自国が占領している都市の数まで行える。

\subsubsection{解体}
既存の建物を故障状態にする。このとき、プレイヤーはターンの開始時に解体したい建物がある地域を占領している必要がある。指定した地域がターン終了時も自国によって占領されている必要は無いが、その地域が他国に占領された場合、もしくは戦闘状態に入った場合は命令は取り消される。\\
命令が実行された場合物資1を消費する。\\
この命令は1ターンで何回でも行える。

\subsubsection{遷都}
主要都市を移設する。このとき、プレイヤーは占領されていない主要都市および包囲状態にない都市がある地域を指定する。移設先となる地域がターン中に他国によって占領された場合は命令は取り消される。なお、戦闘状態に入ったが占領されなかった場合は命令は取り消されない。\\
命令が実行された場合、選択された主要都市は主要都市でなくなり、もう片方が新たな主要都市となる。このとき、現在持っている物資の半分(端数切捨て)を消費する。\\
この命令は1ターンで1回行える。 

\subsubsection{輸出}
他国に物資を輸出する。このとき、プレイヤーは輸出先として選びたい国家と輸出する物資の量を選択する。ターンの終了時、選択した量だけ対象の国家に物資を譲渡する。なお、物資が足りなかった場合譲渡できる分だけ譲渡する。\\
この命令は1ターンで1か国につき1回行える。

\end{document}